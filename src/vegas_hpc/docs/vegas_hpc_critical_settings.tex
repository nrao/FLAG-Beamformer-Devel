% !TEX TS-program = pdflatex
% !TEX encoding = UTF-8 Unicode

% This is a simple template for a LaTeX document using the "article" class.
% See "book", "report", "letter" for other types of document.

\documentclass[11pt]{article} % use larger type; default would be 10pt

\usepackage[utf8]{inputenc} % set input encoding (not needed with XeLaTeX)

%%% Examples of Article customizations
% These packages are optional, depending whether you want the features they provide.
% See the LaTeX Companion or other references for full information.

%%% PAGE DIMENSIONS
\usepackage{geometry} % to change the page dimensions
\geometry{letterpaper} % or letterpaper (US) or a5paper or....
% \geometry{margins=2in} % for example, change the margins to 2 inches all round
% \geometry{landscape} % set up the page for landscape
%   read geometry.pdf for detailed page layout information

\usepackage{graphicx} % support the \includegraphics command and options

% \usepackage[parfill]{parskip} % Activate to begin paragraphs with an empty line rather than an indent

%%% PACKAGES
\usepackage{booktabs} % for much better looking tables
\usepackage{array} % for better arrays (eg matrices) in maths
\usepackage{paralist} % very flexible & customisable lists (eg. enumerate/itemize, etc.)
\usepackage{verbatim} % adds environment for commenting out blocks of text & for better verbatim
\usepackage{subfig} % make it possible to include more than one captioned figure/table in a single float
% These packages are all incorporated in the memoir class to one degree or another...

%%% HEADERS & FOOTERS
\usepackage{fancyhdr} % This should be set AFTER setting up the page geometry
\pagestyle{fancy} % options: empty , plain , fancy
\renewcommand{\headrulewidth}{0pt} % customise the layout...
\lhead{}\chead{}\rhead{}
\lfoot{}\cfoot{\thepage}\rfoot{}

%%% SECTION TITLE APPEARANCE
\usepackage{sectsty}
\allsectionsfont{\sffamily\mdseries\upshape} % (See the fntguide.pdf for font help)
% (This matches ConTeXt defaults)

%%% ToC (table of contents) APPEARANCE
\usepackage[nottoc,notlof,notlot]{tocbibind} % Put the bibliography in the ToC
\usepackage[titles,subfigure]{tocloft} % Alter the style of the Table of Contents
\usepackage{rotating}

\renewcommand{\cftsecfont}{\rmfamily\mdseries\upshape}
\renewcommand{\cftsecpagefont}{\rmfamily\mdseries\upshape} % No bold!

%%% END Article customizations

%%% The "real" document content comes below...

\title{VEGAS: VErsatile Greenbank Astronomical Spectrometer \\ \Large Memo on the Critical Settings for the HPC}
\author{Simon Scott, UC Berkeley}
%\date{} % Activate to display a given date or no date (if empty),
         % otherwise the current date is printed 

\begin{document}
\maketitle
\parskip 7.2pt

\section{Introduction}

This document specifies the settings for the HPC software, for each operating mode. These fields are set via the status shared memory. Table \ref{status-buffer-critical} lists the critical fields that must be set before the VEGAS HPC can be executed.

\begin{table}[!h]
\centering
\caption{Status Buffer Fields for Critical Settings}
\begin{tabular}{l l l}
\hline
\bf Field Name & \bf Data Type & \bf Description \\
\hline
NSUBBAND & Integer & Number of sub-bands (1 to 8) \\
NPOL & Integer & Number of antenna polarisations (must be 2) \\
NCHAN & Integer &  Number of frequency channels per sub-band \\
CHAN\_BW & Double & Width of each spectral channel/bin [Hz]\\
EXPOSURE & Float & Required integration time [s] \\
HWEXPOSR & Float & Hardware integration time (on FPGA or GPU) [s]\\
FPGACLK & Float & FPGA clock rate [Hz] \\
EFSAMPFR & Float & Effective sampling frequency (after decimation) [Hz] \\
SUB0FREQ & Double & Centre frequency of sub-band 0 [Hz] \\
SUB1FREQ & Double & Centre frequency of sub-band 1 [Hz] \\
SUB2FREQ & Double & Centre frequency of sub-band 2 [Hz] \\
SUB3FREQ & Double & Centre frequency of sub-band 3 [Hz] \\
SUB4FREQ & Double & Centre frequency of sub-band 4 [Hz] \\
SUB5FREQ & Double & Centre frequency of sub-band 5 [Hz] \\
SUB6FREQ & Double & Centre frequency of sub-band 6 [Hz] \\
SUB7FREQ & Double & Centre frequency of sub-band 7 [Hz] \\
DATAHOST & String & Hostname of attached ROACH board \\
DATAPORT & Integer & UDP port to which ROACH board transmits packets \\
PKTFMT & String &  Network packet format (must be SPEAD) \\
DATADIR & String & FITS output directory (only when disk thread used) \\
FILENUM & Integer & File number in multi-file scan (reset to zero for each scan) \\
\hline
\end{tabular}
\label{status-buffer-critical}
\end{table}

There are 17 different operating modes, each with its own unique set of parameters. These modes are described in Table \ref{mode-table}. From this point on, each mode will be identified purely by its ``mode number'' (the first column in the table).


\begin{table}[!h]
\centering
\caption{VEGAS Modes}
\begin{tabular}{| p{1.6cm}|  p{4cm} | p{2.4cm} | p{3.5cm} | p{1.6cm} |}
\hline
\bf Mode number & \bf Number of sub-bands per IF [MHz] & \bf Sub-band bandwidth & \bf Number of channels per sub-band & \bf Use GPU? \\
\hline
1 & 1 & 1500 & 1024 & No \\
2 & 1 & 1000 & 2084 & No \\
3 & 1 & 800 & 4096 & No \\
4 & 1 & 500 & 8192 & No \\
5 & 1 & 400 & 16384 & No \\
6 & 1 & 250 & 32768 & Yes \\
7 & 1 & 100 & 32768 & Yes \\
8 & 1 & 50 & 32768 & Yes \\
9 & 1 & 25 & 32768 & Yes \\
10 & 1 & 10 & 32768 & Yes \\
11 & 1 & 5 & 32768 & Yes \\
12 & 1 & 1 & 32768 & Yes \\
13 & 8 & 30 & 4096 & Yes \\
14 & 8 & 15 & 4096 & Yes \\
15 & 8 & 10 & 4096 & Yes \\
16 & 8 & 5 & 4096 & Yes \\
17 & 8 & 1 & 4096 & Yes \\
\hline
\end{tabular}
\label{mode-table}
\end{table}

\section{List of Settings for Each Mode}

Some of the critical settings are mode dependant, while others are mode independant. Table \ref{const-setting-table} shows the recommend values for the mode-independant fields, while Table \ref{mode-setting-table} shows how to set the rest of the fields appropriately for each mode.

\begin{table}[!h]
\centering
\caption{Critical Settings that are Independant of Mode}
\begin{tabular}{| l | l |}
\hline
\bf Field Name & \bf Description \\
\hline
NPOL & Number of polarisations (always 2) \\
EXPOSURE & The required integration time, in seconds (typically between 5e-4 and 60) \\
SUB[0-7]FREQ & The frequency at the centre of the NCHAN/2 frequency bin of each sub-band\\
DATAHOST & The IP address, or host name, of the ROACH board attached to this machine \\
DATAPORT & The network port used by the ROACH (usually 60000) \\
PKTFMT & The network packet format (always "SPEAD") \\
DATADIR & The directory to which the output FITS files should be written \\
FILENUM & Set to zero when starting new observation \\
\hline
\end{tabular}
\label{const-setting-table}
\end{table}

\clearpage

\begin{sidewaystable}[!h]
\centering
\caption{Critical Settings for Each Mode}
\begin{tabular}{| l | l | l | l | l | l | l |}
\hline
\bf Mode & \bf NSUBBAND & \bf NCHAN & \bf CHAN\_BW & \bf HWEXPOSR & \bf FPGACLK & \bf EFSAMPFR \\
\hline
1 & 1 & 1024 & 1.465e6 & 5e-4 & 375e6 & 3e9 \\
2 & 1 & 2084 & 4.88e5 & 5e-4 & 375e6 & 2e9 \\
3 & 1 & 4096 & 1.95e5 & 5e-4 & 375e6 & 1.6e9 \\
4 & 1 & 8192 & 6.1e4 & 5e-4 & 375e6 & 1e9 \\
5 & 1 & 16384 & 2.4e4 & 5e-4 & 375e6 & 800e6 \\
6 & 1 & 32768 & 7.6e3 & 1e-2 & 375e6 & 500e6 \\
7 & 1 & 32768 & 3.1e3 & 1e-2 & 375e6 & 200e6 \\
8 & 1 & 32768 & 1.5e3 & 1e-2 & 375e6 & 100e6 \\
9 & 1 & 32768 & 8e2 & 1e-2 & 375e6 & 50e6 \\
10 & 1 & 32768 & 3e2 & 1e-2 & 375e6 & 20e6 \\
11 & 1 & 32768 & 1.5e2 & 1e-2 & 375e6 & 10e6 \\
12 & 1 & 32768 & 3e1 & 1e-2 & 375e6 & 2e6 \\
13 & 8 & 4096 & 7.3e3 & 1e-2 & 375e6 & 60e6 \\
14 & 8 & 4096 & 3.7e3 & 1e-2 & 375e6 & 30e6 \\
15 & 8 & 4096 & 2.4e3 & 1e-2 & 375e6 & 20e6 \\
16 & 8 & 4096 & 1.2e3 & 1e-2 & 375e6 & 10e6 \\
17 & 8 & 4096 & 2e2 & 1e-2 & 375e6 & 2e6 \\
\hline
\end{tabular}
\label{mode-setting-table}
\end{sidewaystable}

\end{document}